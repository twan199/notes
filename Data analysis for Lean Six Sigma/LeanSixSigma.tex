\documentclass{notes}
\begin{document}
\mytitle{Coursera - Data Analytics for Lean Six Sigma}
\section{DMAIC}
\begin{itemize}[noitemsep]
    \setlength{\itemsep}{0pt}
    \item Define
    \item Measure
    \item Analyze
    \item Improve
    \item Control
\end{itemize}
\subsection{Define}
\textbf{Goal}: Make the problem quantifiable and measurable. How to operationalize the problem. 

80-20 principle: focus on few vital issues instead of many trivial
\subsection{Measure}
\textbf{How}: Find process metrics that are relevant for the project objectives. What characteristics should you measure.
\begin{example}
Mondays process wafer making runs worse than other days? Why is that. What to measure time --> means Duration of start up and Start time production and runs worse --> Amount of scrap per day and volume per day.
\end{example}
These variables are called CTQ\footnote{Critical To Quality characteristic. Also called Y-variable, Dependent variable or Outcome variable}'s: A property of a process or service that is relevant to the project's objective. How to find CTQ's\\
Ask yourself: What metric do I want to change? Will the project be successful if I do that / Does it solve the problem? Examples are processing time\footnote{Time it takes to do a process (often minutes)} and throughput time\footnote{Total duration of processing the change (days)}. Be critical to determination of the CTQ: really contribute to goal?

Unit vs measurement unit.\\
Unit\footnote{Also known as Experimental unit, observational unit, unit of analysis, case, subject}: People, objects, phenomena: the things you collect data on. Example: measure speed highway --> unit is vehicles.\\ 
Measurement unit: defines what a measurement means. The properties of things you collect data on --> km/h

Why sampling is sometimes good idea: 1) cannot measure all data points, 2) when it provides more precise information about a part of the population instead of vague info about all population. Sample has to be representative. 

Data-set: unit in row, variables in columns. Variables are properties that are measure
\subsection{Analyze}
Numerical vs Categorical data.\\
Numerical contains more information.\\
Amount of datapoints needed:
\begin{itemize}
    \item Categorical CTQ: 300
    \item Numerical CTQ: 30
\end{itemize}
Difference in visualizing:
\begin{itemize}
    \item Categorical CTQ: Pie, bar, pareto chart
    \item Numerical CTQ: Histogram, boxplot
\end{itemize}

Before you analyze data: descriptive statistics (mean, se mean, stdev, min, q1, median, q3, max)
\begin{equation}
    Range = max - min
\end{equation}
\begin{equation}
    IQR\footnote{Interquartile range} = q3 - q1 
\end{equation}

\begin{table}[htbp]
    \centering
    \begin{tabular}{l|l l}
    Plot type & x-variable & y-variable (CTQ) \\ \hline
    Scatterplot & Numerical & Numerical \\
    Boxplot & Categorical & Numerical \\
    Boxplot with groups (transposed) & Numerical & Categorical \\
    Stacked bar chart or cross tabulation & Categorical & Categorical \\
    \end{tabular} 
    \caption{Plot type vs variable type}
\end{table}

\begin{table}[htbp]
    \centering
    \begin{tabular}{l|l l}
    Regression type & x-variable & y-variable (CTQ) \\ \hline
    Regression & Numerical & Numerical \\
    ANOVA (Normal dist.) or Kruskal-Wallis (Nonnorm.) & Categorical & Numerical \\
    Logistic regression & Numerical & Categorical \\
    Chi square test & Categorical & Categorical \\
    \end{tabular}
    \caption{Regression type vs variable type}
\end{table}
Few data points --> two sample student t-test can be used.\\
Regression steps:
\begin{enumerate}[noitemsep]
    \item Fitted line plot: Fit a line (linear regression) $y = a + bX$ that describes the relationship
    \item Regression: Is the evidence for this relationship significant
    \item Residuals: Are the residuals normal. Are there any outliers or irregularities
    \item Optional - Prediction interval: Bounds for 95\% of data points
\end{enumerate}

If $p<0.05$ the relationship is significant and $R^2$ should be large. Quadratic regression is $y=a+bX-cX^2$

Instead of normal distribution, for throughput times or processing times, a Weibull distribution can be used or lognormal. 

Probability plot, lowest AD value fits best. Probability plot tells you which distribution to use to model the plot.

Percentages based on counting need at least a sample size of 300.

Empirical CDF can create probability of the population.
\end{document}